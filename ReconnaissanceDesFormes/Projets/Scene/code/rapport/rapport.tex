\documentclass[]{article}   % list options between brackets
\usepackage{amsmath}        % list packages between braces
\usepackage[utf8]{inputenc}
\usepackage{graphicx}
\usepackage{lmodern}

%Page
\addtolength{\oddsidemargin}{-.875in}
\addtolength{\evensidemargin}{-.875in}
\addtolength{\textwidth}{1.75in}
\addtolength{\topmargin}{-.875in}
\addtolength{\textheight}{1.75in}
%Page

% type user-defined commands here
\begin{document}

\title{Études et expérimentation de la classification des scènes naturelles}   % type title between braces
\author{NGUYEN Van Tho - NGUYEN Quoc Khai}         % type author(s) between braces
\date{05 Janvier, 2013}    % type date between braces
\maketitle

\section{Introduction}
Dans ce TP, nous visons à développer un programme qui sert à classifier des scènes naturelles. Concernant ce sujet, il existe plein de recherches. Tout d'abord, ce sont des recherche qui est étudiées pour classifier seulement deux catégories d'images tels que KNN sur les histogrammes des couleurs ou KNN sur les histogrammes des textures. Il existe aussi des méthodes afin de classifier plusieurs de catégories d'image comme Bag of Visual Words, SVM, Pyramid Matche, etc.\\

Pour trouver des bonnes méthodes, nous examinons quelques algorithmes qui rendent des résultats acceptables et qui sont faisables. Après ces études, nous avons choisit la méthodes "Bag of Visual Words".

\section{État de l'art}

\section{Expérimentation}

\subsection{Évaluation}

\section{Analyse des résultats obtenus}

\section{Conclusion}

\begin{thebibliography}{9}
\bibitem{c1}
  Jun Yang, Yu-Gang Jiang, Alexander G. Hauptmann, Chong-Wah Ngo
  \emph{EvaluatingBag-of-Visual-Words RepresentationsClassification}.
 
\end{thebibliography}

\end{document}