\documentclass[a4paper,11pt]{article}
\usepackage{mycv}
\usepackage[utf8]{inputenc}
\name{NGUYEN Van Tho}
\info{Adresse & Institut de la Francophonie pour l'Informatique \\
  & 42 Ta Quang Buu, Hai Ba Trung, Hanoi, Vietnam\\
      Téléphone & +84 (0)9 15 08 31 33\\
      Courriel & nvtho@ifi.edu.vn\\
      }
\begin{document}
\pagenumbering{gobble}
\maketitle
\section{Formation}
\begin{itemFormation}{Depuis 2012}{Master en informatique, option Systèmes Intelligents 
\& Multimédia (SIM)}{Institut de la Francophonie pour l'Informatique, Hanoi, Vietnam}
\end{itemFormation}

\begin{itemFormation}{2011-2012}{Cours intensifs de français}{Institut de la Francophonie 
pour l'Informatique, Hanoi, Vietnam}
\end{itemFormation}

\begin{itemFormation2}{2001-2006}{Diplôme d'ingénieur en informatique}
{option Systèmes d'Information et de Communication}{Institut Polytechnique de Hanoi, Vietnam}
\end{itemFormation2} %add projet de fin d'etudes?

\section{Expérience de recherche}
\begin{itemRecherche} {Octobre 2012 - Juin 2013}{Institut de la Francophonie pour l'Informatique à Hanoi, Vietnam}
\item Projet de recherche Master 1 - 8 ECTS
\item[] Sujet : Intégration de l’émotion dans la simulation d’évacuation en cas d’urgence
\end{itemRecherche}

\section{Expériences professionnelles}

\begin{itemCompany} {Février 2008 - Novembre 2011}{Corporation VNG, Ho Chi Minh, Vietnam}{Ingénieur senior, chef d'équipe de 4 ingénieurs} %senior software engineering
%\item Gestion des projets: Zing Me Mobile (une version du réseau social Zing Me pour téléphones portables), Zing Group (des groupes de discussion qui permettent les utilisateurs de publier des liens, des documents, des questions, des événements, des commentaires. Les groupes sont bien catégorisés en thème ou en géographie)
\item Gestion de projets : Zing Me Mobile (une version du réseau social Zing Me pour 
téléphones portables), Zing Group (des groupes de discussion qui permettent aux 
utilisateurs de publier des liens, des documents, des questions, des événements, des 
commentaires)
\item Analyse des besoins, conception et développement de projets
\item Recherche et évaluation pour choisir de nouvelles technologies : base de données 
noSQL, base de données graphes et système de point d'intérêt
\item Optimisation des systèmes ayant environ 5 millions d'utilisateurs
%\item Optimisation le projet Zing Me Mobile pour les téléphones portables
\end{itemCompany}

\begin{itemCompany} {Août 2006 - Janvier 2008}{Société Vega, Hanoi, Vietnam}{Ingénieur} %senior software engineering
\item Chef de projet pour une application web de gestion de publicités en ligne
\item Configuration et administration de l'application
\item Optimisation de configuration de MySQL, de tables contenant une dizaine de millions 
de lignes
\item Coordination du projet avec l'équipe marketing
\end{itemCompany}

\begin{itemCompany}{Décembre 2005 - Juin 2006}{Corporation FPT, Ha noi, Vietnam}{Stagiaire informatique}
\item Développement d'un framework de système multi-agent basé sur le framework Cougaar
\end{itemCompany}

\section{Compétences}
\begin{tabular}{p{5cm}l}
	Langages & C/C++, Java, Haskell, R, PHP, Bash, Scilab\\
	Bases de données & Mysql, Oracle, PostgreSQL, Cassandra, Neo4J\\
	Systèmes d'exploitation & Linux, Unix, Windows\\
	Outils & Vim, Eclipse, Latex, Subversion, Git
\end{tabular}

\section{Langues}
\begin{tabular}{p{2.5cm}l}
	Vietnamien & Langue maternelle\\
	Français & Niveau B1 (DELF 70/100)\\
	Anglais & Bon niveau technique lu et écrit  %modifier
\end{tabular}
\section{Références}
\begin{itemReference}{M. HO Tuong Vinh}{Directeur de recherche, Responsable de l'option 
SIM}{Institut de la Francophonie pour l'Informatique à Hanoi, 
Vietnam}{ho.tuong.vinh@ifi.edu.vn}
\end{itemReference}
\end{document}